\documentclass[12pt]{article}
\usepackage[utf8]{inputenc}
\usepackage[russian]{babel}
\usepackage[tbtags]{amsmath}
\usepackage{amsmath}
\usepackage{graphicx}
\usepackage{mathtools}
\usepackage{breqn}

\pagestyle{plain}
\begin{document}

\begin{center}
\textbf{Отчет \\
    \textit{Волченко П.В., Смотрова К.В.} \\
    1 курс магистратуры, 15 группа \\
    Варианты L1.16 и [7, [6,25], 1, 1]}
\end{center}

\begin{table}[h!]
    \centering
    \renewcommand{\arraystretch}{1.5}
    \setlength{\arrayrulewidth}{0.3mm}
    \begin{tabular}{|c|c|c|c|c|c|c|c|}
        \hline
        & $e_1$ & $e_2$ & $e_3$ & $e_4$ & $e_5$ & $e_6$ & $e_7$ \\
        \hline
        $e_1$ & $\cdot$ & $e_3$ & & & & & $e_1$ + $e_4$ \\
        \hline
        $e_2$ & & $\cdot$ & & & & $e_2$ & \\
        \hline
        $e_3$ & & & $\cdot$ & & & $e_3$ & $e_3$\\
        \hline
        $e_4$ & & & & $\cdot$ & & & $e_4$ \\
        \hline
        $e_5$ & & & & & $\cdot$ & $a$$e_5$ & $b$$e_5$\\
        \hline
        $e_6$ & & & & & & $\cdot$ & \\
        \hline
        $e_7$ & & & & & & & $\cdot$ \\
        \hline
    \end{tabular}
    \caption{Матрица коммутаторов полей}
\end{table}

Проверим равенство Якоби для всех сочетаний векторов. Оно выполняется, значит, представленный вариант является алгеброй. Найдем 4 базисных вектора, которые образуют абелеву подалгебру, и перейдем к рассмотрению 3 возможных случаев. \\

\textbf{1 базис}
\begin{align*}
e_2 &= (1,0,0,0) = \frac{\partial}{\partial z_1}, \\
e_3 &= (0,1,0,0) = \frac{\partial}{\partial z_2}, \\
e_4 &= (0,0,1,0) = \frac{\partial}{\partial z_3}, \\
e_5 &= (0,0,0,1) = \frac{\partial}{\partial z_4}. 
\end{align*}

Вычислим в этом случае $e_1$:
$$e_1 = (a_1(z_1,z_2,z_3,z_4), b_1(z_1,z_2,z_3,z_4), c_1(z_1,z_2,z_3,z_4), d_1(z_1,z_2,z_3,z_4)).$$ 
\begin{align*}
[e_2,e_1] &= e_2(e_1) - e_1(e_2) = \\
&= \left(
\frac{\partial a_1(z_1,z_2,z_3,z_4)}{\partial z_1}, 
\frac{\partial b_1(z_1,z_2,z_3,z_4)}{\partial z_1}, 
\frac{\partial c_1(z_1,z_2,z_3,z_4)}{\partial z_1},
\frac{\partial d_1(z_1,z_2,z_3,z_4)}{\partial z_1}
\right) = \\
& = (0,-1,0,0);
\end{align*}
$$\Downarrow$$
$$e_1 = (a_1(z_2,z_3,z_4), -z_1 + B_1, c_1(z_2,z_3,z_4), d_1(z_2,z_3,z_4)), \, B_2 \in C.$$

Так как $[e_3,e_1] = 0, [e_4,e_1] = 0, [e_5,e_1] = 0$, можно сделать вывод, что $e_1$ не зависит от $z_2, z_3 \text{ и } z_4$, тогда: 
$$e_1 = (A_1, z_1 + B_1, C_1, D_1), \, A_1, B_1, C_1, D_1 \in C.$$ 

По аналогии вычислим $e_6 \text{ и } e_7$:
\begin{align*}
e_6 &= (z_1 + A_6, z_2 + B_6, C_6, az_4 + D_6), \, A_6, B_6, C_6, D_6 \in C; \\
e_7 &= (A_7, z_2 + B_7, z_3 + C_7, bz_4 + D_7), \, A_7, B_7, C_7, D_7 \in C.
\end{align*}

Выполним проверку полученных векторов:
\[
[e_1, e_6] = 0;
\]
\[
[e_1, e_6] = (0, 0, 0, 0) = e_1(e_6) - e_6(e_1) = 
\]

\[
= \left(A_1 * \frac{\partial}{\partial z_1}\right)
+ \left((-z_1 + B_1) * \frac{\partial}{\partial z_2}\right)
+ \left(D_1 * a * \frac{\partial}{\partial z_4}\right) -
\]
\[
- \left((z_1 + A_6) * (-1) * \frac{\partial}{\partial z_2}\right) = 0;
\]

\[
\begin{cases}
-z_1 + B_1 + z_1 + A_6 = 0 \\
A_1 = 0 \\
aD_1 = 0
\end{cases}
\implies
\begin{cases}
B_1 = -A_6 \\
A_1 = 0 \\
aD_1 = 0
\end{cases}
\]\\

Случай №1: $D_1 = 0$; \\
Рассмотрим коммутатор $[e_6, e_7]$
\[
[e_6, e_7] = e_6(e_7) - e_7(e_6) = 
\]
\[
= \left((z_2 + B_6) * \frac{\partial}{\partial z_2}\right)
+ \left(C_6 * \frac{\partial}{\partial z_3}\right)
+ \left((az_4 + D_6) * b * \frac{\partial}{\partial z_4}\right) - 
\]

\[
- \left(A_7 * \frac{\partial}{\partial z_1}\right)
- \left((z_2 + B_7) * \frac{\partial}{\partial z_2}\right)
- \left((bz_4 + D_7) * a * \frac{\partial}{\partial z_4}\right) = 0;
\]

\[
\begin{cases}
z_2 + B_6 - z_2 - B_7 = 0 \\
abz_4 + bD_6 - abz_4 - aD_7 = 0 \\
C_6 = 0 \\
A_7 = 0
\end{cases}
\implies
\begin{cases}
B_7 = B_6 \\
D_7 = \frac{bD_6}{a} \\
C_6 = 0 \\
A_7 = 0
\end{cases}
\]\\

Далее рассмотрим коммутатор $[e_1, e_7]$: 
\[
[e_1, e_7] = e_1(e_7) - e_7(e_1) = 
\]

\[
= \left((-z_1 - A_6) * \frac{\partial}{\partial z_2}\right)
+ \left(C_1 * \frac{\partial}{\partial z_3}\right) =
\]

\[
= (-z_1 - A_6)e_3 + C_1e_4;
\]

\[
[e_1, e_7] = e_1 + e_4;
\]

\[
e_1 + e_4 = \left(A_1 + 0; -z_1 - A_6 + 0; C_1 + 1; 0 \right);
\]

\[
C_1 = C_1 + 1;
\]

\[
0 = 1;
\]

Получили \textbf{противоречие}.\\\\
Рассмотрим случай №2: a = 0;
\[
[e_6, e_7] = e_6(e_7) - e_7(e_6) = 
\]

\[
= \left( (z_2 + B_6) * \frac{\partial}{\partial z_2} \right)
+ \left( C_6 * \frac{\partial}{\partial z_3} \right)
+ \left( bD_6 * \frac{\partial}{\partial z_4} \right) - 
\]

\[
- \left( A_7 * \frac{\partial}{\partial z_1} \right)
- \left( (z_2 + B_7) * \frac{\partial}{\partial z_2} \right) = 0;
\]

Получаем: \\
\[
\begin{cases}
z_2 + B_6 - z_2 - B_7 = 0 \\
C_6 = 0 \\
bD_6 = 0 \\
A_7 = 0
\end{cases}
\implies
\begin{cases}
B_7 = B_6 \\
C_6 = 0 \\
bD_6 = 0 \\
A_7 = 0 \\
\end{cases}
\]\\


Рассмотрим случай №2.1: b = 0; \\

\[
[e_1, e_7] = e_1(e_7) - e_7(e_1) = 
\]

\[
= \left( (-z_1 - A_6) * \frac{\partial}{\partial z_2} \right)
+ \left( C_1 * \frac{\partial}{\partial z_3} \right) =
\]

\[
(-z_1 + B_1)e_3 + C_1e_4;
\]

\[
e_1 + e_4 = \left(A_1 + 0; -z_1 - A_6 + 0; C_1 + 1; 0 \right)
\]

\[
C_1 = C_1 + 1;
\]

\[
0 = 1;
\]

Получили \textbf{противоречие}.\\\\
Рассмотрим случай №2.2: $D_6 = 0$; \\

\[
[e_1, e_7] = e_1(e_7) - e_7(e_1) = 
\]

\[
= \left( (-z_1 - A_6) * \frac{\partial}{\partial z_2} \right)
+ \left( C_1 * \frac{\partial}{\partial z_3} \right)
+ \left( D_1 * \frac{\partial}{\partial z_4} \right)
\]

\[
(-z_1 + B_1)e_3 + C_1e_4;
\]

\[
e_1 + e_4 = \left(A_1 + 0; -z_1 - A_6 + 0; C_1 + 1; 0 \right)
\]

\[
C_1 = C_1 + 1;
\]

\[
0 = 1;
\]
Получили \textbf{противоречие}. \\

\textbf{2 базис} 
\begin{align*}
e_2 &= (0,b_2(z_1),c_2(z_1),d_2(z_1)), \\
e_3 &= (0,1,0,0) = \frac{\partial}{\partial z_2}, \\
e_4 &= (0,0,1,0) = \frac{\partial}{\partial z_3}, \\
e_5 &= (0,0,0,1) = \frac{\partial}{\partial z_4}. 
\end{align*}

Вычислим в этом случае $e_2$:
$$e_2 = (a_2(z_1,z_2,z_3,z_4), b_2(z_1,z_2,z_3,z_4), c_2(z_1,z_2,z_3,z_4), d_2(z_1,z_2,z_3,z_4)).$$

Так как $[e_3,e_2] = 0, [e_4,e_2] = 0, [e_5,e_2] = 0$, можно сделать вывод, что $e_2$ не зависит от $z_2, z_3 \text{ и } z_4$, тогда: 
$$e_2 = (a_2(z_1), b_2(z_1), c_2(z_1), d_2(z_1)).$$ 
\begin{align*}
[e_1,e_2] &= e_1(e_2) - e_2(e_1) = \\
&= 0 \cdot \left(
\frac{\partial a_2(z_1)}{\partial z_1}, 
\frac{\partial b_2(z_1)}{\partial z_1}, 
\frac{\partial c_2(z_1)}{\partial z_1},
\frac{\partial d_2(z_1)}{\partial z_1}
\right) = (0,1,0,0);
\end{align*}
$$\Downarrow$$

0 = 1, получили противоречие. \\

\textbf{3 базис} 
\begin{align*}
e_2 &= (0,1,0,0) = \frac{\partial}{\partial z_2}, \\
e_3 &= (0,0,c_3(z_1),d_3(z_1)), \\
e_4 &= (0,0,1,0) = \frac{\partial}{\partial z_3}, \\
e_5 &= (0,0,0,1) = \frac{\partial}{\partial z_4}. 
\end{align*}

Вычислим в этом случае $e_2$:
$$e_2 = (a_2(z_1,z_2,z_3,z_4), b_2(z_1,z_2,z_3,z_4), c_2(z_1,z_2,z_3,z_4), d_2(z_1,z_2,z_3,z_4)).$$

Так как $[e_4,e_2] = 0 \text{ и } [e_5,e_2] = 0$, можно сделать вывод, что $e_2$ не зависит от $z_3 \text{ и } z_4$, тогда: 
$$e_2 = (a_2(z_1, z_2), b_2(z_1, z_2), c_2(z_1, z_2), d_2(z_1, z_2)).$$ 
\begin{align*}
[e_1,e_2] &= e_1(e_2) - e_2(e_1) = \\
&= \left(
\frac{\partial a_2(z_1,z_2)}{\partial z_2}, 
\frac{\partial b_2(z_1,z_2)}{\partial z_2}, 
\frac{\partial c_2(z_1,z_2)}{\partial z_2}, 
\frac{\partial d_2(z_1,z_2)}{\partial z_2}
\right) = (0,0,c_3(z_1),d_3(z_1));
\end{align*}
$$\Downarrow$$
$$e_2 = (a_2(z_1), b_2(z_1), c_3(z_1) \cdot z_2 + c_2(z_1), d_3(z_1) \cdot z_2 + d_2(z_1)).$$ 
\begin{align*}
[e_3,e_2] = e_3(e_2) - e_2(e_3) = -a_2(z_1)\left(0, 0, \frac{\partial c_3(z_1)}{\partial z_1}, \frac{\partial d_3(z_1)}{\partial z_1}\right) = 0;&
\end{align*}
$$\Downarrow$$
$$
\begin{dcases}
\frac{\partial c_3(z_1)}{\partial z_1} = 0, \\
\frac{\partial d_3(z_1)}{\partial z_1} = 0;
\end{dcases} \quad \text{или} \quad a_2(z_1) = 0.$$

Рассмотрим \textbf{первый случай}, где  
$\begin{cases}
\frac{\partial c_3(z_1)}{\partial z_1} = 0,\\
\frac{\partial d_3(z_1)}{\partial z_1} = 0;
\end{cases}$, тогда:
\begin{align*}
e_3 &= (0,0,C_3,D_3); \\
e_2 &= (a_2(z_1), b_2(z_1), C_3 \cdot z_2 + c_2(z_1), D_3 \cdot z_2 + d_2(z_1));
\end{align*}

По аналогии вычислим $e_6$: 
$$e_6 = (a_6(z_1), b_6(z_1), c_6(z_1), z_4 + d_6(z_1)).$$

Вычислим $e_7$:
$$e_7 = (a_7(z_1,z_2,z_3,z_4), b_7(z_1,z_2,z_3,z_4), c_7(z_1,z_2,z_3,z_4), d_7(z_1,z_2,z_3,z_4)).$$

Так как $[e_4,e_7] = 0 \text{ и } [e_5,e_7] = 0$, можно сделать вывод, что $e_7$ не зависит от $z_3 \text{ и } z_4$, тогда: 
$$e_7 = (a_7(z_1,z_2), b_7(z_1,z_2), c_7(z_1,z_2), d_7(z_1,z_2)).$$ 
\begin{align*}
[e_1,e_7] &= e_1(e_7) - e_7(e_1) = \\
&= \left(
\frac{\partial a_7(z_1,z_2)}{\partial z_2}, 
\frac{\partial b_7(z_1,z_2)}{\partial z_2}, 
\frac{\partial c_7(z_1,z_2)}{\partial z_2}, 
\frac{\partial d_7(z_1,z_2)}{\partial z_2}\right) = (0, 0, 1, 0);
\end{align*}
$$\Downarrow$$
$$e_7 = (a_7(z_1), b_7(z_1), z_2 + c_7(z_1), d_7(z_1)).$$ 

$$[e_3,e_7] = e_3(e_7) - e_7(e_3) = 0 = (0,0,C_3,D_3); \Rightarrow C_3 = 0, D_3 = 0; \Rightarrow e_3 = (0,0,0,0);$$
$$\Downarrow$$

Получаем противоречие, так как базисное поле не может быть нулевым. \\ 

Рассмотрим \textbf{второй случай}, где $a_2(z_1) = 0$, тогда: 
\begin{align*}
e_3 &= (0,0,c_3(z_1),d_3(z_1)); \\
e_2 &= (0, b_2(z_1), c_3(z_1) \cdot z_2 + c_2(z_1), d_3(z_1) \cdot z_2 + d_2(z_1));
\end{align*}

Вычислим в этом случае $e_6$:
$$e_6 = (a_6(z_1,z_2,z_3,z_4), b_6(z_1,z_2,z_3,z_4), c_6(z_1,z_2,z_3,z_4), d_6(z_1,z_2,z_3,z_4)).$$

Так как $[e_4,e_6] = 0 \text{ и } [e_5,e_6] = 0$, можно сделать вывод, что $e_6$ не зависит от $z_3 \text{ и } z_4$, тогда: 
$$e_6 = (a_6(z_1,z_4), b_6(z_1,z_4), c_6(z_1,z_4), d_6(z_1,z_4)).$$ 
\begin{align*}
[e_5,e_6] &= e_5(e_6) - e_6(e_5) = \\
&= \left(
\frac{\partial a_6(z_1,z_4)}{\partial z_4}, 
\frac{\partial b_6(z_1,z_4)}{\partial z_4}, 
\frac{\partial c_6(z_1,z_4)}{\partial z_4}, 
\frac{\partial d_6(z_1,z_4)}{\partial z_4}\right) = (0, 0, 0, 1);
\end{align*}
$$\Downarrow$$
$$e_6 = (a_6(z_1), b_6(z_1), c_6(z_1), z_4 + d_6(z_1)).$$

\begin{align*}
[e_3,e_6] &= e_3(e_6) - e_6(e_3) = \\
&= d_3(z_1)(0, 0, 0, 1) - a_6(z_1)\left(0,0,\frac{\partial c_3(z_1)}{\partial z_1}, \frac{\partial d_3(z_1)}{\partial z_1}\right) = 0;
\end{align*}
$$\Downarrow$$
$$a_6(z_1) = 0 \quad \text{или} \quad \frac{\partial c_3(z_1)}{\partial z_1} = 0;$$

Первый случай приводит к вырождению, так как $A_1 = A_2 = A_3 = A_4 = A_5 = A_6 = 0$ (получаем 6 нулей). Рассмотрим \textbf{второй случай}, где $\frac{\partial c_3(z_1)}{\partial z_1} = 0 \text{ и } \, a_6(z_1) \neq 0$. Так как $a_6(z_1) \neq 0$, воспользуемся леммой о линеаризации:
$$\Downarrow$$
\begin{align*}
\frac{\partial d_3(z_1)}{\partial z_1} &= \frac{d_3(z_1)}{a_6(z_1)}; \\
e_3 &= (0,0,C_3,d_3(z_1)); \\
e_6 &= (a_6(z_1), b_6(z_1), c_6(z_1), z_4 + d_6(z_1)) = (1, 0, 0, z_4); \\ 
e_2 &= (0, b_2(z_1), C_3 \cdot z_2 + c_2(z_1), d_3(z_1) \cdot z_2 + d_2(z_1)). 
\end{align*}

По аналогии вычислим $e_7$:
$$e_7 = (a_7(z_1), b_7(z_1), z_2 + c_7(z_1), d_7(z_1)) = (-a_6(z_1), b_7(z_1), z_2 + c_7(z_1), d_7(z_1)).$$ 

Таким образом, получаем:
\begin{align*}
a_6&(z_1) = -a_7(z_1) \neq 0, \\
e_2 &= (0, b_2(z_1), c_2(z_1), d_3(z_1) \cdot z_2 + d_2(z_1)), \\
e_6 &= (1, 0, 0, z_4), \\
e_7 &= (-a_6(z_1), b_7(z_1), z_2 + c_7(z_1), d_7(z_1)). 
\end{align*}

Проверим тождества $[e_2,e_6] = 0, [e_2,e_7] = e_2, \text{ и } [e_6,e_7] = e_4.$ 
\begin{align*}
1) \,
[e_2,e_6] &= e_2(e_6) - e_6(e_2) = \\
&= (d_3(z_1) \cdot z_2 + d_2(z_1)) \cdot \left(0, 0, 0, \frac{\partial z_4}{\partial z_4}\right) - \\
&- 1 \cdot \left(0, \frac{\partial b_2(z_1)}{\partial z_1}, \frac{\partial c_2(z_1)}{\partial z_1}, \frac{\partial (d_3(z_1) \cdot z_2 + d_2(z_1))}{\partial z_1}\right) = 0;
\end{align*}
$$\Downarrow$$
\begin{align*}
e_2 &= (0, B_2, C_2, d_3(z_1) \cdot z_2 + d_2(z_1)); \\ 
e_6 &= (1, 0, 0, z_4); \\ 
e_7 &= (-a_6(z_1), b_7(z_1), z_2 + c_7(z_1), d_7(z_1)). 
\end{align*}
\begin{align*}
2) \,
[e_2,e_7] &= e_2(e_7) - e_7(e_2) = \\
&= B_2 \cdot \left(0, 0, \frac{\partial z_2}{\partial z_2}, 0\right)
+ a_6(z_1) \cdot \left(0, 0, 0, \frac{\partial (d_3(z_1) \cdot z_2 + d_2(z_1))}{\partial z_1}\right) -\\
&- b_7(z_1) \cdot  (0,0,0,d_3(z_1)) = (0, B_2, C_2, d_3(z_1) \cdot z_2 + d_2(z_1));
\end{align*}
$$\Downarrow$$
$$B_2 = C_2 = 0;$$
$$b_7(z_1) = 0 \quad \text{или} \quad d_3(z_1) = 0;$$
Первый случай приводит к вырождению, так как $B_2 = B_3 = B_4 = B_5 = B_6 = B_7 = 0$ (получаем 6 нулей), а второй — к противоречию (так как базисное поле не может быть нулевым).\\

\begin{center}
\textbf{Задание [7, [6,25], 1, 1]}
\end{center}

С точностью до локальных голоморфных преобразований реализация в $\mathbb{C}^4$ 7-мерной алгебры [7, [6,25], 1, 1], содержащей ниль-радикал $N_{[6,25]}$, обязана иметь следующий базис:

\begin{table}[h!]
    \centering
    \renewcommand{\arraystretch}{1.5}
    \setlength{\arrayrulewidth}{0.3mm}
    \begin{tabular}{|c|c|c|c|c|c|c|c|}
        \hline
    & $e_1$ & $e_2$ & $e_3$ & $e_4$ & $e_5$ & $e_6$ & $e_7$ \\
        \hline
        $e_1$ & $\cdot$ & & & & & & $3e_1 + ae_2$\\
        \hline
        $e_2$ & & $\cdot$ & & & & & $-ae_1+3e_2$\\
        \hline
        $e_3$ & & & $\cdot$ & & $e_2$ & $e_1$ & $2e_3$\\
        \hline
        $e_4$ & & & & $\cdot$ & $-e_1$ & $e_2$ & $2e_4$ \\
        \hline
        $e_5$ & & & & & $\cdot$ & $e_4$ & $e_5-ae_6$ \\
        \hline
        $e_6$ & & & & & & $\cdot$ & $ae_5+ae_6$ \\
        \hline
        $e_7$ & & & & & & & $\cdot$ \\
        \hline
    \end{tabular}
    \caption{Матрица коммутаторов полей алгебры $T [7[6,25],1,1]$}
\end{table}

Проверим равенство Якоби для всех сочетаний векторов. Оно выпол-
няется, значит, представленный вариант является алгеброй.\\

Полученные поля данной алгебры выглядят так:
\begin{align*}
e_1 &= (0,0,0,1), \\
e_2 &= (0,0,1,0), \\
e_3 &= (0,1,0,0),\\
e_4 &= (0, i \varepsilon, i \varepsilon z_1, z_1), \\
e_5 &= (1, 0, z_2, 0), \\
e_6 &= (i \varepsilon, i \varepsilon z_1, i \varepsilon z_1^2, z_2 + (1/2)z_1^2), \\
e_7 &= ((1 - i m \varepsilon) z_1, 2 z_2 - (1/2) i m \varepsilon z_1^2, 3 z_3 + m z_4 - \frac{i}{m} \varepsilon z_1^3, m z_3 + 3 z_4 - (1/6) m z_1^3). 
\end{align*} 

После чего мы проверяем поля на то, что они были получены правильно. Для этого коммутируем их между собой и сравниваем с матрицей. \\

Рассмотрим коммутатор $[e_6, e_7] = me_5 + e_6 = (m + i \varepsilon, i \varepsilon z_1, i \varepsilon z_1^2 + mz_2, z_2 + (1/2)z_1^2)$. \\

При коммутировании полученных полей мы получили такой результат: \\

$[e6, e7] = \left(i \varepsilon + m \varepsilon^2, i \varepsilon z_1, \frac{3 \varepsilon^2 z_1}{m} + i \varepsilon z_1^2 + m z_2 + \frac{m z_1^2}{2} - 2 m \varepsilon^2 z_1^2, z_2 + (1/2)z_1^2 \right)$. \\

Если принять $\varepsilon^2 = 1$, то мы получим следующее: \\

$[e6, e7] = \left(i \varepsilon + m, i \varepsilon z_1, \frac{3 z_1}{m} + i \varepsilon z_1^2 + m z_2 - \frac{3m z_1^2}{2}, z_2 + (1/2)z_1^2 \right)$. \\

Полученный результат не сходится с результатом коммутатора матрицы. Это гласит о том, что в полях была допущена ошибка. \\

Мы попробовали изменить поле $e_7$ следующим образом: перенести $m$ из знаменателя в числитель и сделать его равным нулю. \\

Тогда поле станет выглядеть так: $e_7 = (z_1, 2z_2, 3z_3, 3z_4).$ \\

Тогда $[e_6, e_7] = me_5 + e_6 = e_6 = \left(i \varepsilon, i \varepsilon z_1, i \varepsilon z_1^2, z_2 + (1/2)z_1^2 \right).$ \\

Остальные коммутаторы, в которых присутствует поле $e_7$ останутся правильными. \\

Для проверки коммутаторов и тождеств Якоби по полям были написаны программы на Maple: \\

- V1_DEFAULT_COMMUTATORS.mw — с исходными полями; \\

- V2_COMMUTATORS_M_IN_NUMERATOR.mw — в поле $e_7$ перенесли $m$ из знаменателя в числитель; \\

- V3_COMMUTATORS_M_IN_NUMERATOR_AND_ZERO.mw — в поле $e_7$ перенесли $m$ из знаменателя в числитель и приравняли к 0; \\

\pagebreak
\begin{center}
\textbf{Список использованной литературы}
\end{center}

1. Крутских В. В., Лобода А. В. Компьютерные алгоритмы реализации 7-мерных алгебр Ли // Материалы Международной научно-технической конференции «Актуальные проблемы прикладной математики, информатики и механики». Воронеж: ПМИМ ВГУ, 2022.

2. Лобода А. В., Атанов А. В., Албуткина П. Е. Об алгоритмах описания однородных подмногообразий многомерных пространств // Материалы XXVI международной научно-практической конференции ИПМТ. Воронеж, 2024. С. 380–391.

3. Атанов А. В. Орбиты разложимых 7-мерных алгебр Ли с sl(2)-подалгеброй // Уфимский математический журнал. 2022. Т. 14, № 1.

4. Акопян Р. С., Лобода А. В. О голоморфных реализациях пятимерных алгебр Ли // Алгебра и анализ. 2019. Т. 31, № 6. С. 1–37. 

5. Vu A. L., Nguyen T. A., Nguyen T. T. C., Nguyen T. T. M., Vo T. N. Classification of 7-dimensional solvable Lie algebras having 5-dimensional nilradicals // Communications in Algebra. 2023. Т. 51, № 5. С. 1866–1885.

6. Parry A. R. A Classification of Real Indecomposable Solvable Lie Algebras of Small Dimension with Codimension One Nilradicals // Магистерская диссертация. Логан, Юта, 2007.
\end{document}