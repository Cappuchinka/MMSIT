\href{\href{}{}}{}\documentclass[14pt]{extarticle} % Подключаем расширенный класс с поддержкой 14pt
\usepackage[T2A]{fontenc}      % Кодировка для кириллицы
\usepackage[utf8]{inputenc}     % Кодировка ввода UTF-8
\usepackage[russian]{babel}     % Пакет для русского языка
\usepackage{amsmath}
\usepackage{array}
\usepackage{amsmath, amssymb} % Пакеты для математических символов
\usepackage[a4paper, margin=1in]{geometry} % Настройка полей
\setlength{\parindent}{0pt} % Убирает отступ в начале абзаца

\usepackage{tocloft}
\usepackage{titlesec}
\usepackage{hyperref}
\hypersetup{
    colorlinks=false, % Отключаем цвет ссылок
    hidelinks,        % Убираем рамки и цвета, делаем ссылки как обычный текст
    pdfborder={0 0 0} % Полностью убираем рамки вокруг ссылок
}

\renewcommand{\contentsname}{Оглавление}

\setlength{\cftbeforesecskip}{0pt}     % Убирает отступы между разделами
\setlength{\cftbeforesubsecskip}{0pt}  % Убирает отступы между подразделами
\setcounter{tocdepth}{2}

% Настройка стиля заголовков
\titleformat{\section}
  {\normalfont\Large\bfseries}{\thesection}{1em}{}
\titleformat{\subsection}
  {\normalfont\large\bfseries}{\thesubsection}{1em}{}

% Настройка точек в содержании для разделов и подразделов
\renewcommand{\cftsecleader}{\cftdotfill{\cftdotsep}}     % Для разделов
\renewcommand{\cftsubsecleader}{\cftdotfill{\cftdotsep}}  % Для подразделов

% Установка обычного шрифта для разделов в содержании
\renewcommand{\cftsecfont}{\normalfont}        % Обычный шрифт для разделов
\renewcommand{\cftsecpagefont}{\normalfont}    % Обычный шрифт для номеров страниц разделов


\begin{document}

\linespread{1,5}
% Первая страница (без номера)
\thispagestyle{empty} % Убираем номер на первой странице
\begin{titlepage}
    \newpage
    \begin{center}
    {\bfseries Министерство образования и науки Российской Федерации \\
    ВГУ}
    \vspace{1cm}

     Кафедра Технологий обработки и защиты информации
     \vspace{2em}

     \end{center}

    \vspace{5em}

    \begin{center}
    \Large
\textbf{Задание по дисциплине}

\normalsize
\textbf{«Математические Методы в Современных Информационных Технологиях»}

(1 курс магистратуры ИСиТ, СПИИ, 2024-2025 уч. год.)

Лобода А.В., Каверина В.К.
     \end{center}
    \vspace{5em}


   \vbox{%
\hfill%
\vbox{%
\hbox{\textbf{Выполнили:} магистранты 1 курса}%
\hbox{направления «Информационные системы и технологии»}%
\hbox{профиля «Системы прикладного искусственного интеллекта»}%
\hbox{группы 15}%
\hbox{Волченко Полина Владимировна}%
\hbox{Смотрова Кристина Владимировна}%
\hbox{}%
}%
}

    \begin{center}
    \vspace{10em}
    2024
    \end{center}

    \end{titlepage}
\newpage

% Добавляем содержание
\tableofcontents
\newpage

\section{Постановка задачи}

Задание L1.16. \\

Необходимо найти невырожденные 7-мерные орбиты в пространстве $ \mathbb{C}^4 $ (или показать их отсутствие) у 7-мерной алгебры Ли. В положительном случае, проинтегрировать алгебру Ли.

\begin{table}[h!]
    \centering
    \renewcommand{\arraystretch}{1.5}
    \setlength{\arrayrulewidth}{0.3mm}
    \begin{tabular}{|c|c|c|c|c|c|c|c|}
        \hline
        & $e_1$ & $e_2$ & $e_3$ & $e_4$ & $e_5$ & $e_6$ & $e_7$ \\
        \hline
        $e_1$ & & $e_3$ & & & & & $e_1$ + $e_4$ \\
        \hline
        $e_2$ & & & & & & $e_2$ & \\
        \hline
        $e_3$ & & & & & & $e_3$ & $e_3$\\
        \hline
        $e_4$ & & & & & & & $e_4$ \\
        \hline
        $e_5$ & & & & & & $a$$e_5$ & $b$$e_5$\\
        \hline
        $e_6$ & & & & & & & \\
        \hline
        $e_7$ & & & & & & & \\
        \hline
    \end{tabular}
    \caption{Матрица коммутаторов полей}
\end{table}

\newpage
\section{Проверка алгебры}

Проверка абелевых алгебр условием Якоби необходима для определения интегрируемости системы уравнений. Тождества Якоби для структурных констант являются необходимым и достаточным условием интегрируемости этих уравнений. \\

Для проверки возьмем комбинацию \(e_1, e_2\) и \(e_3:\) \\

[[e_1, e_2], e_3] + [[e_3, e_1], e_2] + [[e_2, e_3], e_1] = 0 \\

1)
\[
 [[e_1, e_2], e_6]: \\
\]
\[
[e_1, e_2] = e_3
\]
\[
[e_3, e_3] = 0
\] \\

2)
\[
[[e_3, e_1], e_2]: \\
\]
\[
[e_3, e_1] = 0
\]
\[
[0, e_2] = 0
\] \\

3)
\[
[[e_2, e_3], e_1]: \\
\]
\[
[e_2, e_3] = 0
\]
\[
[0, e_1] = 0
\]

\[
0 + 0 + 0 = 0
\]

В следствии чего можем сделать вывод, что Тождеcтво Якоби выполняется. \\

\\
Дальнейшие проверки, оставшихся 34-х комбинаций, были сделаны с помощью программы на языке Python.
\\

В результате было получено, что все комбинации удовлетворяют условию Якоби.
\newpage
\section{Решение}
Согласно лемме~[1], голоморфной заменой координат пространства $ \mathbb{C}^4 $ базис
\( (e_2, e_3, e_4, e_5) \) из четырёх полей, образующих 4-мерную абелеву подалгебру, можно привести к одному из трёх видов:

\begin{enumerate}
    \item
    \[
    \begin{aligned}
    e_1 &: (1, 0, 0, 0), \\
    e_3 &: (0, 1, 0, 0), \\
    e_4 &: (0, 0, 1, 0), \\
    e_5 &: (0, 0, 0, 1).
    \end{aligned}
    \]

    \item
    \[
    \begin{aligned}
    e_2 &: (0, b_2(z_1), c_2(z_1), d_2(z_1)), \\
    e_3 &: (0, 1, 0, 0), \\
    e_4 &: (0, 0, 1, 0), \\
    e_5 &: (0, 0, 0, 1).
    \end{aligned}
    \]

    \item
    \[
    \begin{aligned}
    e_2 &: (1, 0, 0, 0), \\
    e_3 &: (0, 0, c_3(z_1), d_3(z_1)), \\
    e_4 &: (0, 0, 1, 0), \\
    e_5 &: (0, 0, 0, 1).
    \end{aligned}
    \]
\end{enumerate}

\subsection{Первый случай}

Рассмотрим 1-й случай. Поля \( e_2, e_3, e_4, e_5 \) образуют абелеву подалгебру, так как их коммутаторы попарно между собой дают ноль. \\

Поля \( e_2, e_3, e_4, e_5 \) выглядят следующим образом:
\[
e_2 = (1, 0, 0, 0) = \frac{\partial}{\partial z_1},
\]
\[
e_3 = (0, 1, 0, 0) = \frac{\partial}{\partial z_2},
\]
\[
e_4 = (0, 0, 1, 0) = \frac{\partial}{\partial z_3},
\]
\[
e_5 = (0, 0, 0, 1) = \frac{\partial}{\partial z_4}.
\]
Для нахождения остальных полей (\( e_1, e_6, e_7 \)), рассмотрим коммутаторы между ними и полями из абелевой подалгебры, а также между собой. \\

Сначала найдем поле $e_1$.\\
\[
[e_2, e_1] = -e_3;
\]
\[
[e_3, e_1] = 0;
\]
\[
[e_4, e_1] = 0;
\]
\[
[e_5, e_1] = 0;
\]\\
Рассмотрим коммутатор $[e_2, e_1]$:\\
\[
[e_2, e_1] = -e_3 = (0, -1, 0, 0) = e_2(e_1) - e_1(e_2) =
\]
\[
= 0 - e_1(e_2) = - e_1(e_2) = - \frac{\partial }{\partial z_2};
\]

\[
\frac{\partial b_1}{\partial z_1} = -1 \implies b_1(z) = -z_1 + B_1.
\]

В остальных случаях поле $e_1$ не зависит от переменных $z_2, z_3, z_4$.\\
Поле \( e_1 \) имеет следующий вид:
\[
\boldsymbol{e_1 = \big(A_1; -z_1 + B_1; C_1; D_1\big)}.
\]
По аналогии мы получим следующие коммутаторы:
\[
\boldsymbol{e_6 = (z_1 + A_6; z_2 + B_6; C_6; az_4 + D_6)}.
\]
\[
\boldsymbol{e_7 = \big(A_7; z_2 + B_7; z_3 + C_7; bz_4 + D_7\big)}.
\]
Общий вид будет такой:
\[
e_1 = \big( A_1; -z_1 + B_1; C_1; D_1 \big)
\]
\[
e_2 = \big( 1; 0; 0; 0 \big)
\]
\[
e_3 = \big( 0; 1; 0; 0 \big)
\]
\[
e_4 = \big( 0; 0; 1; 0 \big)
\]
\[
e_5 = \big( 0; 0; 0; 1 \big)
\]
\[
e_6 = \big( z_1 + A_6; z_2 + B_6; C_6; az_4 + D_6 \big)
\]
\[
e_7 = \big( A_7; z_2 + B_7; z_3 + C_7; bz4 + D_7 \big)
\]
Далее рассмотрим следующий коммутатор.
\[
[e_1, e_6] = 0;
\]
\[
[e_1, e_6] = (0, 0, 0, 0) = e_1(e_6) - e_6(e_1) =
\]

\[
= \left(A_1 * \frac{\partial}{\partial z_1}\right)
+ \left((-z_1 + B_1) * \frac{\partial}{\partial z_2}\right)
+ \left(D_1 * a * \frac{\partial}{\partial z_4}\right) -
\]
\[
- \left((z_1 + A_6) * (-1) * \frac{\partial}{\partial z_2}\right) = 0;
\]

\[
\begin{cases}
-z_1 + B_1 + z_1 + A_6 = 0 \\
A_1 = 0 \\
aD_1 = 0
\end{cases}
\implies
\begin{cases}
B_1 = -A_6 \\
A_1 = 0 \\
aD_1 = 0
\end{cases}
\]\\

Случай №1: $D_1 = 0$; \\
Рассмотрим коммутатор $[e_6, e_7]$
\[
[e_6, e_7] = e_6(e_7) - e_7(e_6) =
\]
\[
= \left((z_2 + B_6) * \frac{\partial}{\partial z_2}\right)
+ \left(C_6 * \frac{\partial}{\partial z_3}\right)
+ \left((az_4 + D_6) * b * \frac{\partial}{\partial z_4}\right) -
\]

\[
- \left(A_7 * \frac{\partial}{\partial z_1}\right)
- \left((z_2 + B_7) * \frac{\partial}{\partial z_2}\right)
- \left((bz_4 + D_7) * a * \frac{\partial}{\partial z_4}\right) = 0;
\]

\[
\begin{cases}
z_2 + B_6 - z_2 - B_7 = 0 \\
abz_4 + bD_6 - abz_4 - aD_7 = 0 \\
C_6 = 0 \\
A_7 = 0
\end{cases}
\implies
\begin{cases}
B_7 = B_6 \\
D_7 = \frac{bD_6}{a} \\
C_6 = 0 \\
A_7 = 0
\end{cases}
\]\\

Далее рассмотрим коммутатор $[e_1, e_7]$:
\[
[e_1, e_7] = e_1(e_7) - e_7(e_1) =
\]

\[
= \left((-z_1 - A_6) * \frac{\partial}{\partial z_2}\right)
+ \left(C_1 * \frac{\partial}{\partial z_3}\right) =
\]

\[
= (-z_1 - A_6)e_3 + C_1e_4;
\]

\[
[e_1, e_7] = e_1 + e_4;
\]

\[
e_1 + e_4 = \left(A_1 + 0; -z_1 - A_6 + 0; C_1 + 1; 0 \right);
\]

\[
C_1 = C_1 + 1;
\]

\[
0 = 1;
\]

Получили \textbf{противоречие}.\\\\
Рассмотрим случай №2: a = 0;
\[
[e_6, e_7] = e_6(e_7) - e_7(e_6) =
\]

\[
= \left( (z_2 + B_6) * \frac{\partial}{\partial z_2} \right)
+ \left( C_6 * \frac{\partial}{\partial z_3} \right)
+ \left( bD_6 * \frac{\partial}{\partial z_4} \right) -
\]

\[
- \left( A_7 * \frac{\partial}{\partial z_1} \right)
- \left( (z_2 + B_7) * \frac{\partial}{\partial z_2} \right) = 0;
\]

Получаем: \\
\[
\begin{cases}
z_2 + B_6 - z_2 - B_7 = 0 \\
C_6 = 0 \\
bD_6 = 0 \\
A_7 = 0
\end{cases}
\implies
\begin{cases}
B_7 = B_6 \\
C_6 = 0 \\
bD_6 = 0 \\
A_7 = 0 \\
\end{cases}
\]\\


Рассмотрим случай №2.1: b = 0; \\

\[
[e_1, e_7] = e_1(e_7) - e_7(e_1) =
\]

\[
= \left( (-z_1 - A_6) * \frac{\partial}{\partial z_2} \right)
+ \left( C_1 * \frac{\partial}{\partial z_3} \right) =
\]

\[
(-z_1 + B_1)e_3 + C_1e_4;
\]

\[
e_1 + e_4 = \left(A_1 + 0; -z_1 - A_6 + 0; C_1 + 1; 0 \right)
\]

\[
C_1 = C_1 + 1;
\]

\[
0 = 1;
\]

Получили \textbf{противоречие}.\\\\
Рассмотрим случай №2.2: $D_6 = 0$; \\

\[
[e_1, e_7] = e_1(e_7) - e_7(e_1) =
\]

\[
= \left( (-z_1 - A_6) * \frac{\partial}{\partial z_2} \right)
+ \left( C_1 * \frac{\partial}{\partial z_3} \right)
+ \left( D_1 * \frac{\partial}{\partial z_4} \right)
\]

\[
(-z_1 + B_1)e_3 + C_1e_4;
\]

\[
e_1 + e_4 = \left(A_1 + 0; -z_1 - A_6 + 0; C_1 + 1; 0 \right)
\]

\[
C_1 = C_1 + 1;
\]

\[
0 = 1;
\]
Получили \textbf{противоречие}.
\subsection{Второй случай}

Рассмотрим 2-й случай. Поля \( e_1, e_3, e_4, e_5 \) образуют абелеву подалгебру, так как их коммутаторы попарно между собой дают ноль. \\

Поля \( e_2, e_3, e_4, e_5 \) выглядят следующим образом:
\[
e_2 = (0, b_2(z_1), c_2(z_1), d_2(z_1)),
\]
\[
e_3 = (0, 1, 0, 0),
\]
\[
e_4 = (0, 0, 1, 0),
\]
\[
e_5 = (0, 0, 0, 1).
\]

Для нахождения остальных полей (\( e_1, e_6, e_7 \)), рассмотрим коммутаторы между ними и полями из абелевой подалгебры, а также между собой. \\

В результате чего мы получили такие поля:
\[
\boldsymbol{e_1 = \big( A_1e^{-z_1};\ B_1;\ -z_1 + C_1;\ D_1e^{(b - 1)z_1}\big);}
\]
\[
\boldsymbol{e_2 = \big( 0; \frac{e^{z_1}}{A_1}; 0; 0\big);}
\]
\[
\boldsymbol{e_3 = \big( 0; 1;\ 0;\ 0 \big);}
\]
\[
\boldsymbol{e_4 = \big( 0; 0; 1; 0 \big);}
\]
\[
\boldsymbol{e_5 = \big( 0; 0; 0; 1 \big);}
\]
\[
\boldsymbol{e_6 = \big( 0;\ B_6;\ C_6;\ D_6 \big);}
\]
\[
\boldsymbol{e_7 = \big( 1;\ z_2;\ z_3;\ bz_4 \big);}
\]\\
Как можно заметить, в данном случае получаем \textbf{вырождение}.

\subsection{Третий случай}

Рассмотрим 3-й случай. Поля \( e_2, e_3, e_4, e_5 \) образуют абелеву подалгебру, так как их коммутаторы попарно между собой дают ноль. \\

Поля \( e_2, e_3, e_4, e_5 \) выглядят следующим образом:
\[
e_2 = (0, 1, 0, 0),
\]
\[
e_3 = (0, 0, c_3(z_1), d_3(z_1)),
\]
\[
e_4 = (0, 0, 1, 0),
\]
\[
e_5 = (0, 0, 0, 1).
\]
Для нахождения остальных полей (\( e_1, e_6, e_7 \)), рассмотрим коммутаторы между ними и полями из абелевой подалгебры, а также между собой. \\

В результате получаем следующие поля:\\

\[
\boldsymbol{e_1 = \big( 0;\ B_1e^{z_1};\ C_1;\  -D_3e^{(a - 1)z_1}z_2 + D_1e^{-az_1}\big);}
\]
\[
\boldsymbol{e_2 = \big( 0; 1; 0; 0\big);}
\]
\[
\boldsymbol{e_3 = \big( 0; 0;\ 0;\ D_3e^{(a - 1)z_1} \big);}
\]
\[
\boldsymbol{e_4 = \big( 0; 0; 1; 0 \big);}
\]
\[
\boldsymbol{e_5 = \big( 0; 0; 0; 1 \big);}
\]
\[
\boldsymbol{e_6 = \big( 1;\ z_2;\ 0;\ az_4 \big);}
\]
\[
\boldsymbol{e_7 = \big( A_7;\ B_7e^{z_1};\ C_7;\ bz_4 + D_7e^{az_1} \big);}
\]

Как можно заметить, в данном случае также получаем \textbf{вырождение}.

\newpage
\section{Интегрирование алгебры $T [7, [6, 25], 1, 1]$}

Предложение 11. \textit{С точностью до локальных голоморфных преобразований лобальной реализации алгебры $T [7, [6, 25], 1, 1]$ в пространстве $\mathbb{C}^4$, удовлетворяющая условиям $A$ и $B$, имеет базис $\{e\}$}

\begin{table}[h!]
    \centering
    \renewcommand{\arraystretch}{1.5}
    \setlength{\arrayrulewidth}{0.3mm}
    \begin{tabular}{|c|c|c|c|c|c|c|c|}
        \hline
    & $e_1$ & $e_2$ & $e_3$ & $e_4$ & $e_5$ & $e_6$ & $e_7$ \\
        \hline
        $e_1$ & & & & & & & $3e_1 + ae_2$\\
        \hline
        $e_2$ & & & & & & & $-ae_1+3e_2$\\
        \hline
        $e_3$ & & & & & $e_2$ & $e_1$ & $2e_3$\\
        \hline
        $e_4$ & & & & & $-e_1$ & $e_2$ & $2e_4$ \\
        \hline
        $e_5$ & & & & & & $e_4$ & $e_5-ae_6$ \\
        \hline
        $e_6$ & & & & & & & $ae_5+ae_6$ \\
        \hline
        $e_7$ & & & & & & & \\
        \hline
    \end{tabular}
    \caption{Матрица коммутаторов полей алгебры $T [7[6,20],1,1]$}
\end{table}



\subsection{Проверка алгебры}

Для проверки возьмем комбинацию \(e_1, e_2\) и \(e_6:\) \\

[[e_1, e_2], e_6] + [[e_2, e_6], e_1] + [[e_6, e_1], e_2] = 0 \\

1)
\[
 [[e_1, e_2], e_6]: \\
\]
\[
[e_1, e_2] = 0
\]
\[
[0, e_6] = 0
\] \\

2)
\[
[[e_2, e_6], e_1]: \\
\]
\[
[e_2, e_6] = 0
\]
\[
[0, e_6] = 0
\] \\

3)
\[
[[e_6, e_1], e_2]: \\
\]
\[
[e_6, e_1] = 0
\]
\[
[0, e_2] = 0
\]

Из полученных решений получаем следующий итог:
\[
0 + 0 + 0 = 0
\]

В следствии чего можем сделать вывод, что Тождество Якоби выполняется. \\
\\
Дальнейшие проверки оставшихся 34-х комбинаций были сделаны с помощью программы на языке Python.
\\

В результате было получено, что все комбинации удовлетворяют условию Якоби.

\subsection{Решение}

Искомая функция от восьми вещественных \(\Phi(x_1,x_2,x_3,x_4,y_1,y_2,y_3,y_4)\), должна подчиняться условию[5]:\newline
Вещественная гиперповерхность
\(M=\lbrace{\Phi=0}\rbrace\) является орбитой
(интегральной поверхностью) голоморфной реализации алгебры Ли \(g\) ,
если для каждого базисного поля этой алгебры выполняется условие
касания \(M\) в виде
\[
\mathfrak{Re}\lbrace{e_k(\Phi)|_M\rbrace}=0
\]\newline

Также важна формула:
\[
\frac{\partial\Phi}{\partial z_k}=\frac{1}{2}(\frac{\partial\Phi}{\partial x_k}- i\frac{\partial\Phi}{\partial y_k})
\]\newline

Поля алгебры Ли  [7, [6, 25],1, 1]: \\

$e_1 = (0, 0, 0, 1)$, \\

$e_2 = (0, 0, 1, 0)$, \\

$e_3 = (0, 1, 0, 0)$, \\

$e_4 = (0, i \varepsilon, i \varepsilon z_1, z_1)$, \\

$e_5 = (1, 0, z_2, 0)$, \\

$e_6 = \left( i \varepsilon, i \varepsilon z_1, i \varepsilon z_1^2, z_2 + (1/2)z_1^2 \right)$, \\

$e_7 = \left( (1 - i m \varepsilon) z_1, 2 z_2 - (1/2) i m \varepsilon z_1^2, 3 z_3 + m z_4 - \frac{i}{m} \varepsilon z_1^3, m z_3 + 3 z_4 - (1/6) m z_1^3 \right)$.\\

Варинта решения будет представлен программным решением среды Maple. Исходный код и решеним представлено в приложении B.

\newpage
\phantomsection % Создает точку привязки для правильной работы ссылок
\addcontentsline{toc}{section}{Список использованных источников}
\begin{center}
    \Large \textbf{Список использованных источников}
\end{center}

\begin{enumerate}
    \item https://edu.vsu.ru/mod/resource/view.php?id=1009551
    \item Parry А.R. A Classification of Real Indecomposable Solvable Lie Algebras of Small Dimension with Codimension One Nilradicals// Master’s thesis "--- Logan, Utah, 2007

    \item {\it Vu~A. L., Nguyen~T. A., Nguyen~T. T. C., Nguyen~T. T. M., Vo~T. N.} Classification of 7-dimensional solvable Lie algebras having 5-dimensional nilradicals// Communications in Algebra. 2023. Vol. 51, № 5. P. 1866--1885.

\end{enumerate}
\newpage
\begin{center}
    \Large \textbf{Приложение А}
\end{center}
\newpage
\begin{center}
    \Large \textbf{Приложение В}
\end{center}

\end{document}
