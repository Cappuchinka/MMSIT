\documentclass[12pt]{article}
\usepackage[utf8]{inputenc}
\usepackage[russian]{babel}
\usepackage[tbtags]{amsmath}
\usepackage{amsmath}
\usepackage{graphicx}
\usepackage{mathtools}
\usepackage{breqn}

\pagestyle{plain}
\begin{document}
\linespread{1,5}
% Первая страница (без номера)
\thispagestyle{empty} % Убираем номер на первой странице
\begin{titlepage}
    \newpage
    \begin{center}
    {\bfseries Министерство образования и науки Российской Федерации \\
    ВГУ}
    \vspace{1cm}

     Кафедра Технологий обработки и защиты информации
     \vspace{2em}

     \end{center}

    \vspace{5em}

    \begin{center}
    \Large
\textbf{Задание по дисциплине}

\normalsize
\textbf{«Математические Методы в Современных Информационных Технологиях»}

(1 курс магистратуры ИСиТ, СПИИ, 2024-2025 уч. год.)

Лобода А.В., Каверина В.К.
     \end{center}
    \vspace{5em}


   \vbox{%
\hfill%
\vbox{%
\hbox{\textbf{Выполнили:} магистранты 1 курса}%
\hbox{направления «Информационные системы и технологии»}%
\hbox{профиля «Системы прикладного искусственного интеллекта»}%
\hbox{группы 15}%
\hbox{Волченко Полина Владимировна}%
\hbox{Смотрова Кристина Владимировна}%
\hbox{}%
}%
} 

    \begin{center}
    \vspace{5em}
    2024
    \end{center}

    \end{titlepage}
\newpage

\section{Алгебра L1.16}

\begin{table}[h!]
    \centering
    \renewcommand{\arraystretch}{1.5}
    \setlength{\arrayrulewidth}{0.3mm}
    \begin{tabular}{|c|c|c|c|c|c|c|c|}
        \hline
        & $e_1$ & $e_2$ & $e_3$ & $e_4$ & $e_5$ & $e_6$ & $e_7$ \\
        \hline
        $e_1$ & $\cdot$ & $e_3$ & & & & & $e_1$ + $e_4$ \\
        \hline
        $e_2$ & & $\cdot$ & & & & $e_2$ & \\
        \hline
        $e_3$ & & & $\cdot$ & & & $e_3$ & $e_3$\\
        \hline
        $e_4$ & & & & $\cdot$ & & & $e_4$ \\
        \hline
        $e_5$ & & & & & $\cdot$ & $a$$e_5$ & $b$$e_5$\\
        \hline
        $e_6$ & & & & & & $\cdot$ & \\
        \hline
        $e_7$ & & & & & & & $\cdot$ \\
        \hline
    \end{tabular}
    \caption{Матрица коммутаторов полей}
\end{table}

Проверим равенство Якоби для всех сочетаний векторов. \\

Для проверки возьмем комбинацию \(e_1, e_2\) и \(e_3:\) 

$$[[e_1, e_2], e_3] + [[e_3, e_1], e_2] + [[e_2, e_3], e_1] = 0$ 

1)
\[
 [[e_1, e_2], e_3]: \\
\]
\[
[e_1, e_2] = e_3 
\]
\[
[e_3, e_3] = 0 
\] 

2)
\[
[[e_3, e_1], e_2]: \\
\]
\[
[e_3, e_1] = 0 
\]
\[
[0, e_2] = 0 
\] 

3) 
\[
[[e_2, e_3], e_1]: \\
\]
\[
[e_2, e_3] = 0 
\]
\[
[0, e_1] = 0 
\]

\[
0 + 0 + 0 = 0
\]

В следствии чего можем сделать вывод, что Тождеcтво Якоби выполняется. \\

\\
Оно выполняется, значит, представленный вариант является алгеброй. Найдем 4 базисных вектора, которые образуют абелеву подалгебру, и перейдем к рассмотрению 3 возможных случаев. 
\\

\subsection{Первый базис}
\begin{align*}
e_2 &= (1,0,0,0) = \frac{\partial}{\partial z_1}, \\
e_3 &= (0,1,0,0) = \frac{\partial}{\partial z_2}, \\
e_4 &= (0,0,1,0) = \frac{\partial}{\partial z_3}, \\
e_5 &= (0,0,0,1) = \frac{\partial}{\partial z_4}. 
\end{align*}

Вычислим в этом случае $e_1$:
$$e_1 = (a_1(z_1,z_2,z_3,z_4), b_1(z_1,z_2,z_3,z_4), c_1(z_1,z_2,z_3,z_4), d_1(z_1,z_2,z_3,z_4)).$$ 
\begin{align*}
[e_2,e_1] &= e_2(e_1) - e_1(e_2) = \\
&= \left(
\frac{\partial a_1(z_1,z_2,z_3,z_4)}{\partial z_1}, 
\frac{\partial b_1(z_1,z_2,z_3,z_4)}{\partial z_1}, 
\frac{\partial c_1(z_1,z_2,z_3,z_4)}{\partial z_1},
\frac{\partial d_1(z_1,z_2,z_3,z_4)}{\partial z_1}
\right) = \\
& = (0,-1,0,0);
\end{align*}
$$\Downarrow$$
$$e_1 = (a_1(z_2,z_3,z_4), -z_1 + B_1, c_1(z_2,z_3,z_4), d_1(z_2,z_3,z_4)), \, B_2 \in C.$$

Так как $[e_3,e_1] = 0, [e_4,e_1] = 0, [e_5,e_1] = 0$, можно сделать вывод, что $e_1$ не зависит от $z_2, z_3 \text{ и } z_4$, тогда: 
$$e_1 = (A_1, z_1 + B_1, C_1, D_1), \, A_1, B_1, C_1, D_1 \in C.$$ 

По аналогии вычислим $e_6 \text{ и } e_7$:
\begin{align*}
e_6 &= (z_1 + A_6, z_2 + B_6, C_6, az_4 + D_6), \, A_6, B_6, C_6, D_6 \in C; \\
e_7 &= (A_7, z_2 + B_7, z_3 + C_7, bz_4 + D_7), \, A_7, B_7, C_7, D_7 \in C.
\end{align*}

Выполним проверку полученных векторов:
\[
[e_1, e_6] = 0;
\]
\[
[e_1, e_6] = (0, 0, 0, 0) = e_1(e_6) - e_6(e_1) = 
\]

\[
= \left(A_1 * \frac{\partial}{\partial z_1}\right)
+ \left((-z_1 + B_1) * \frac{\partial}{\partial z_2}\right)
+ \left(D_1 * a * \frac{\partial}{\partial z_4}\right) -
\]
\[
- \left((z_1 + A_6) * (-1) * \frac{\partial}{\partial z_2}\right) = 0;
\]

\[
\begin{cases}
-z_1 + B_1 + z_1 + A_6 = 0 \\
A_1 = 0 \\
aD_1 = 0
\end{cases}
\implies
\begin{cases}
B_1 = -A_6 \\
A_1 = 0 \\
aD_1 = 0
\end{cases}
\]\\

Случай №1: $D_1 = 0$; \\
Рассмотрим коммутатор $[e_6, e_7]$
\[
[e_6, e_7] = e_6(e_7) - e_7(e_6) = 
\]
\[
= \left((z_2 + B_6) * \frac{\partial}{\partial z_2}\right)
+ \left(C_6 * \frac{\partial}{\partial z_3}\right)
+ \left((az_4 + D_6) * b * \frac{\partial}{\partial z_4}\right) - 
\]

\[
- \left(A_7 * \frac{\partial}{\partial z_1}\right)
- \left((z_2 + B_7) * \frac{\partial}{\partial z_2}\right)
- \left((bz_4 + D_7) * a * \frac{\partial}{\partial z_4}\right) = 0;
\]

\[
\begin{cases}
z_2 + B_6 - z_2 - B_7 = 0 \\
abz_4 + bD_6 - abz_4 - aD_7 = 0 \\
C_6 = 0 \\
A_7 = 0
\end{cases}
\implies
\begin{cases}
B_7 = B_6 \\
D_7 = \frac{bD_6}{a} \\
C_6 = 0 \\
A_7 = 0
\end{cases}
\]\\

Далее рассмотрим коммутатор $[e_1, e_7]$: 
\[
[e_1, e_7] = e_1(e_7) - e_7(e_1) = 
\]

\[
= \left((-z_1 - A_6) * \frac{\partial}{\partial z_2}\right)
+ \left(C_1 * \frac{\partial}{\partial z_3}\right) =
\]

\[
= (-z_1 - A_6)e_3 + C_1e_4;
\]

\[
[e_1, e_7] = e_1 + e_4;
\]

\[
e_1 + e_4 = \left(A_1 + 0; -z_1 - A_6 + 0; C_1 + 1; 0 \right);
\]

\[
C_1 = C_1 + 1;
\]

\[
0 = 1;
\]

Получили \textbf{противоречие}.\\\\
Рассмотрим случай №2: a = 0;
\[
[e_6, e_7] = e_6(e_7) - e_7(e_6) = 
\]

\[
= \left( (z_2 + B_6) * \frac{\partial}{\partial z_2} \right)
+ \left( C_6 * \frac{\partial}{\partial z_3} \right)
+ \left( bD_6 * \frac{\partial}{\partial z_4} \right) - 
\]

\[
- \left( A_7 * \frac{\partial}{\partial z_1} \right)
- \left( (z_2 + B_7) * \frac{\partial}{\partial z_2} \right) = 0;
\]

Получаем: \\
\[
\begin{cases}
z_2 + B_6 - z_2 - B_7 = 0 \\
C_6 = 0 \\
bD_6 = 0 \\
A_7 = 0
\end{cases}
\implies
\begin{cases}
B_7 = B_6 \\
C_6 = 0 \\
bD_6 = 0 \\
A_7 = 0 \\
\end{cases}
\]\\


Рассмотрим случай №2.1: b = 0; \\

\[
[e_1, e_7] = e_1(e_7) - e_7(e_1) = 
\]

\[
= \left( (-z_1 - A_6) * \frac{\partial}{\partial z_2} \right)
+ \left( C_1 * \frac{\partial}{\partial z_3} \right) =
\]

\[
(-z_1 + B_1)e_3 + C_1e_4;
\]

\[
e_1 + e_4 = \left(A_1 + 0; -z_1 - A_6 + 0; C_1 + 1; 0 \right)
\]

\[
C_1 = C_1 + 1;
\]

\[
0 = 1;
\]

Получили \textbf{противоречие}.\\\\
Рассмотрим случай №2.2: $D_6 = 0$; \\

\[
[e_1, e_7] = e_1(e_7) - e_7(e_1) = 
\]

\[
= \left( (-z_1 - A_6) * \frac{\partial}{\partial z_2} \right)
+ \left( C_1 * \frac{\partial}{\partial z_3} \right)
+ \left( D_1 * \frac{\partial}{\partial z_4} \right)
\]

\[
(-z_1 + B_1)e_3 + C_1e_4;
\]

\[
e_1 + e_4 = \left(A_1 + 0; -z_1 - A_6 + 0; C_1 + 1; 0 \right)
\]

\[
C_1 = C_1 + 1;
\]

\[
0 = 1;
\]
Получили \textbf{противоречие}. 

\subsection{Второй базис} 
\begin{align*}
e_2 &= (0,b_2(z_1),c_2(z_1),d_2(z_1)), \\
e_3 &= (0,1,0,0), \\
e_4 &= (0,0,1,0), \\
e_5 &= (0,0,0,1). 
\end{align*}

\textbf{Вычислим $e_1$:}
$$e_1 = (a_1(z_1,z_2,z_3,z_4), b_1(z_1,z_2,z_3,z_4), c_1(z_1,z_2,z_3,z_4), d_1(z_1,z_2,z_3,z_4)).$$

Так как $[e_1,e_3] = 0, [e_1,e_4] = 0, [e_1,e_5] = 0$, можно сделать вывод, что $e_1$ не зависит от $z_2, z_3 \text{ и } z_4$, тогда: 
$$e_1 = (a_1(z_1), b_1(z_1), c_1(z_1), d_1(z_1)).$$ 

Рассмотрим $[e_1,e_2] = e_3$ \\

Отсюда мы получаем $\frac{\partial{c_2(z_1)}}{\partial{z_1}} = 0$, $\frac{\partial{d_2(z_1)}}{\partial{z_1}} = 0$ и $a_1(z_1) \cdot \frac{\partial{b_2(z_1)}}{\partial{z_1}} = 1$. \\

Это означает, что $c_2(z_1) = C_2$, $d_2(z_1) = D_2$, а $a_1(z_1) \neq 0$. \\

\textbf{Вычислим $e_7$:}
$$e_7 = (a_7(z_1,z_2,z_3,z_4), b_7(z_1,z_2,z_3,z_4), c_7(z_1,z_2,z_3,z_4), d_7(z_1,z_2,z_3,z_4)).$$

При рассмотрении коммутационных соотношений $[e_3, e_7], [e_4, e_7], [e_6, e_7]$ поле $e_7$ примет следующий вид: 
$$e_7 = (a_7(z_1), z_2 + b_7(z_1), z_3 + c_7(z_1), bz_4 + d_7(z_1).$$ 

Линеаризируем поле $e_7$: \\

$e_7 = (1, z_2, z_3 + bz_4)$. \\

Рассмотрим $[e_2,e_7] = 0.\\

Отсюда мы получаем $b_2(z_1) = e^{z_1} \cdot B_2$.\\

\textbf{Вычислим $e_6$:}
$$e_6 = (a_6(z_1,z_2,z_3,z_4), b_6(z_1,z_2,z_3,z_4), c_6(z_1,z_2,z_3,z_4), d_6(z_1,z_2,z_3,z_4)).$$

Так как $[e_4,e_6] = 0$, можно сделать вывод, что $e_6$ не зависит от $z_3$, тогда: 
$$e_6 = (a_6(z_1,z_2,z_4), b_6(z_1,z_2,z_4), c_6(z_1,z_2,z_4), d_6(z_1,z_2,z_4)).$$ 

Рассмотрим $[e_5,e_6] = a \cdot e_5 \text{ и } [e_3,e_6] = e_3.\\

Отсюда мы получаем $\frac{\partial d_6 }{\partial z_4} = a \text{ и } \frac{\partial b_6 }{\partial z_2} = 1, \text{ а остальное = 0. } $\\

Рассмотрим $[e_2, e_6] = e_2$. \\

Отсюда мы получаем: $a_6(z_1) \cdot \frac{\partial{b_2(z_1)}}{\partial{z_1}} = 0$, $-a_6(z_1) \cdot \frac{\partial{c_2(z_1)}}{\partial{z_1}} = c_2(z_1)$ и $ad_2(z_1) - a_6(z_1) \cdot \frac{\partial{b_2(z_1)}}{\partial{z_1}} = d_2(z_1)$. \\

Тогда $e_6$:
$$e_6 = (a_6(z_1), b_6(z_1) + z_2, c_6(z_1), d_6(z_1) + az_4).$$

После данных вычислений мы стали коммутировать поля $(e_1, e_6, e_7)$ между собой и получили такие поля:
\[
\boldsymbol{e_1 = \big( A_1e^{-z_1};\ B_1;\ -z_1 + C_1;\ D_1e^{(b - 1)z_1}\big);}
\]
\[
\boldsymbol{e_2 = \big( 0; \frac{e^{z_1}}{A_1}; 0; 0\big);}
\]
\[
\boldsymbol{e_3 = \big( 0; 1;\ 0;\ 0 \big);}
\]
\[
\boldsymbol{e_4 = \big( 0; 0; 1; 0 \big);}
\]
\[
\boldsymbol{e_5 = \big( 0; 0; 0; 1 \big);}
\]
\[
\boldsymbol{e_6 = \big( 0;\ B_6;\ C_6;\ D_6 \big);}
\]
\[
\boldsymbol{e_7 = \big( 1;\ z_2;\ z_3;\ bz_4 \big);}
\]\\
Обращая внимание на поля $(e_2, e_3)$ мы замечаем, что тут у нас \textbf{вырождение}.

\subsection{Третий базис} 
\begin{align*}
e_2 &= (0,1,0,0), \\
e_3 &= (0,0,c_3(z_1),d_3(z_1)), \\
e_4 &= (0,0,1,0), \\
e_5 &= (0,0,0,1). 
\end{align*}

\textbf{Вычислим $e_1$:}
$$e_1 = (a_1(z_1,z_2,z_3,z_4), b_1(z_1,z_2,z_3,z_4), c_1(z_1,z_2,z_3,z_4), d_1(z_1,z_2,z_3,z_4)).$$

Так как $[e_4,e_1] = 0, [e_5,e_1] = 0$, можно сделать вывод, что $e_1$ не зависит от $z_3 \text{ и } z_4$, тогда: 
$$e_1 = (a_1(z_1,z_2), b_1(z_1,z_2), c_1(z_1,z_2), d_1(z_1,z_2)).$$ 

Рассмотрим $[e_2,e_1] = -e_3$ \\

Отсюда мы получаем $\frac{\partial a_1 }{\partial z_2} = 0 \text{ и } \frac{\partial b_1 }{\partial z_2} = 0 \text{ и } \frac{\partial c_1 }{\partial z_2} = -c_3(z_1) \text{ и } \frac{\partial d_1 }{\partial z_2} = -d_3(z_1)$\\

Тогда $e_1$:
$$e_1 = (a_1(z_1), b_1(z_1), -c_3(z_1) \cdot  z_2, -d_3(z_1)\cdot  z_2).$$

Рассмотрим $[e_3,e_1] = 0$

\begin{align*}
[e_3,e_1] &= e_3(e_1) - e_1(e_3) = \\
&= 0 - a_1(z_1) \cdot \left(
\frac{\partial c_3}{\partial z_1} \cdot \frac{\partial}{\partial z_3}, 
\frac{\partial c_3}{\partial z_1} \cdot \frac{\partial}{\partial z_4}
\right) = 0;
\end{align*}

Если $a_1(z_1) = 0$,  \text{ то } $$e_1 = (0, b_1(z_1), -c_3(z_1) \cdot  z_2, -d_3(z_1)\cdot  z_2).$$

Если $\left(
\frac{\partial c_3}{\partial z_1} \cdot \frac{\partial}{\partial z_3}, 
\frac{\partial c_3}{\partial z_1} \cdot \frac{\partial}{\partial z_4}
\right) = 0$,  \text{ то } $$ \frac{\partial c_3(z_1)}{\partial z_1} = 0 => c_3(z_1)=C_3 \text{ и } \frac{\partial d_3(z_1)}{\partial z_1} = 0 => d_3(z_1)=D_3. \text{ Тогда } e_3 = (0, 0, C_3 , D_3). $$ 
\\

\textbf{Вычислим $e_7$:}
$$e_7 = (a_7(z_1,z_2,z_3,z_4), b_7(z_1,z_2,z_3,z_4), c_7(z_1,z_2,z_3,z_4), d_7(z_1,z_2,z_3,z_4)).$$

Так как $[e_2,e_7] = 0$, можно сделать вывод, что $e_7$ не зависит от $z_2$, тогда: 
$$e_7 = (a_7(z_1,z_3,z_4), b_7(z_1,z_3,z_4), c_7(z_1,z_3,z_4), d_7(z_1,z_3,z_4)).$$ 

Рассмотрим $[e_5,e_7] = b \cdot e_5 \text{ и } [e_4,e_7] = e_4.\\

Отсюда мы получаем $\frac{\partial d_7 }{\partial z_4} = b \text{ и } \frac{\partial c_7 }{\partial z_3} = 1, \text{ а остальное = 0. } $\\

Тогда $e_7$:
$$e_7 = (a_7(z_1), b_7(z_1), c_7(z_1) + z_3, d_7(z_1) + b \cdot z_4).$$

При рассмотрение $[e_3,e_7] = e_3$ получаем: 
\begin{equation*}
 \begin{cases}
   $$a_7(z_1) \cdot \frac{\partial c_3(z_1) }{\partial z_1} = 0$$ \\
   $$d_3(z_1) = b \cdot d_3(z_1) - a_7(z_1) \frac{\partial d_3(z_1) }{\partial z_1}$$
 \end{cases}
\end{equation*}

Из $$a_7(z_1) \cdot \frac{\partial c_3(z_1) }{\partial z_1} = 0$$ следует, что $$a_7(z_1) = 0 => b = 1 \text{ или } \frac{\partial c_3(z_1) }{\partial z_1} = 0 => c_3(z_1)=C_3. \text{ Тогда } e_3 = (0, 0, C_3, d_3(z_1)). $$ 

\textbf{Вычислим $e_6$:}
$$e_6 = (a_6(z_1,z_2,z_3,z_4), b_6(z_1,z_2,z_3,z_4), c_6(z_1,z_2,z_3,z_4), d_6(z_1,z_2,z_3,z_4)).$$

Так как $[e_4,e_6] = 0$, можно сделать вывод, что $e_6$ не зависит от $z_3$, тогда: 
$$e_6 = (a_6(z_1,z_2,z_4), b_6(z_1,z_2,z_4), c_6(z_1,z_2,z_4), d_6(z_1,z_2,z_4)).$$ 

Рассмотрим $[e_5,e_6] = a \cdot e_5 \text{ и } [e_2,e_6] = e_2.\\

Отсюда мы получаем $\frac{\partial d_6 }{\partial z_4} = a \text{ и } \frac{\partial b_6 }{\partial z_2} = 1, \text{ а остальное = 0. } $\\

Тогда $e_6$:
$$e_6 = (a_6(z_1), b_6(z_1) + z_2, c_6(z_1), d_6(z_1) + a \cdot z_4).$$

При рассмотрение $[e_3,e_6] = e_3$ получаем: 
\begin{equation*}
 \begin{cases}
   $$c_3(z_1) + a_6(z_1) \cdot \frac{\partial c_3(z_1) }{\partial z_1} = 0$$ \\
   $$d_3(z_1) + a_6(z_1) \cdot d_3(z_1) - a \cdot d_3(z_1)$$
 \end{cases}
\end{equation*}

После данных вычислений мы стали коммутировать поля $(e_1, e_6, e_7)$ между собой и получили такие поля:


\[
\boldsymbol{e_1 = \big( 0;\ B_1 \cdot e^{z_1};\ C_1;\ -e^{(a-1)z_1}\cdot D_3 \cdot z_2 + e^{-az_1} \cdot D_1\big);}
\]
\[
\boldsymbol{e_2 = \big( 0; 1; 0; 0\big);}
\]
\[
\boldsymbol{e_3 = \big( 0; 0;\ 0;\ D_3\cdot e^{(a-1)z_1} \big);}
\]
\[
\boldsymbol{e_4 = \big( 0; 0; 1; 0 \big);}
\]
\[
\boldsymbol{e_5 = \big( 0; 0; 0; 1 \big);}
\]
\[
\boldsymbol{e_6 = \big( 1;\ z_2;\ 0;\ a \cdot z_4 \big);}
\]
\[
\boldsymbol{e_7 = \big( A_7;\ B_7 \cdot e^{z_1};\ C_7;\ bz_4 + D_7 \cdot e^{az_1} \big);}
\]\\
Обращая внимание на поля $(e_3, e_5)$ мы замечаем, что тут у нас \textbf{вырождение}.
\pagebreak

\section{Задание [7, [6,25], 1, 1]}

С точностью до локальных голоморфных преобразований реализация в $\mathbb{C}^4$ 7-мерной алгебры [7, [6,25], 1, 1], содержащей ниль-радикал $N_{[6,25]}$, обязана иметь следующий базис:

\begin{table}[h!]
    \centering
    \renewcommand{\arraystretch}{1.5}
    \setlength{\arrayrulewidth}{0.3mm}
    \begin{tabular}{|c|c|c|c|c|c|c|c|}
        \hline
    & $e_1$ & $e_2$ & $e_3$ & $e_4$ & $e_5$ & $e_6$ & $e_7$ \\
        \hline
        $e_1$ & $\cdot$ & & & & & & $3e_1 + me_2$\\
        \hline
        $e_2$ & & $\cdot$ & & & & & $-me_1+3e_2$\\
        \hline
        $e_3$ & & & $\cdot$ & & $e_2$ & $e_1$ & $2e_3$\\
        \hline
        $e_4$ & & & & $\cdot$ & $-e_1$ & $e_2$ & $2e_4$ \\
        \hline
        $e_5$ & & & & & $\cdot$ & $e_4$ & $e_5-me_6$ \\
        \hline
        $e_6$ & & & & & & $\cdot$ & $me_5+e_6$ \\
        \hline
        $e_7$ & & & & & & & $\cdot$ \\
        \hline
    \end{tabular}
    \caption{Матрица коммутаторов полей алгебры $T [7[6,25],1,1]$}
\end{table}

Проверим равенство Якоби для всех сочетаний векторов. Оно выполняется, значит, представленный вариант является алгеброй. \\

Для проверки возьмем комбинацию \(e_1, e_2\) и \(e_6:\) \\

$[[e_1, e_2], e_6] + [[e_2, e_6], e_1] + [[e_6, e_1], e_2] = 0$ \\

1)
\[
 [[e_1, e_2], e_6]: \\
\]
\[
[e_1, e_2] = 0
\]
\[
[0, e_6] = 0 
\] \\

2)
\[
[[e_2, e_6], e_1]: \\
\]
\[
[e_2, e_6] = 0
\]
\[
[0, e_6] = 0 
\] \\

3) 
\[
[[e_6, e_1], e_2]: \\
\]
\[
[e_6, e_1] = 0 
\]
\[
[0, e_2] = 0 
\]

Из полученных решений получаем следующий итог:
\[
0 + 0 + 0 = 0
\]

В следствии чего можем сделать вывод, что Тождество Якоби выполняется. \\

Данные нам поля этой алгебры выглядят так:
\begin{align*}
e_1 &= (0,0,0,1), \\
e_2 &= (0,0,1,0), \\
e_3 &= (0,1,0,0),\\
e_4 &= (0, i \varepsilon, i \varepsilon z_1, z_1), \\
e_5 &= (1, 0, z_2, 0), \\
e_6 &= (i \varepsilon, i \varepsilon z_1, i \varepsilon z_1^2, z_2 + (1/2)z_1^2), \\
e_7 &= ((1 - i m \varepsilon) z_1, 2 z_2 - (1/2) i m \varepsilon z_1^2, 3 z_3 + m z_4 - \frac{i}{m} \varepsilon z_1^3, m z_3 + 3 z_4 - (1/6) m z_1^3). 
\end{align*} 

После чего мы проверяем поля на то, что они были получены правильно. Для этого коммутируем их между собой и сравниваем с матрицей. \\

Рассмотрим коммутатор $[e_6, e_7] = me_5 + e_6 = (m + i \varepsilon, i \varepsilon z_1, i \varepsilon z_1^2 + mz_2, z_2 + (1/2)z_1^2)$. \\

При коммутировании полученных полей мы получили такой результат: \\

$[e6, e7] = \left(i \varepsilon + m \varepsilon^2, i \varepsilon z_1, \frac{3 \varepsilon^2 z_1}{m} + i \varepsilon z_1^2 + m z_2 + \frac{m z_1^2}{2} - 2 m \varepsilon^2 z_1^2, z_2 + (1/2)z_1^2 \right)$. \\

Если принять $\varepsilon^2 = 1$, то мы получим следующее: \\

$[e6, e7] = \left(i \varepsilon + m, i \varepsilon z_1, \frac{3 z_1}{m} + i \varepsilon z_1^2 + m z_2 - \frac{3m z_1^2}{2}, z_2 + (1/2)z_1^2 \right)$. \\

Полученный результат не сходится с результатом коммутатора матрицы. Это гласит о том, что в полях была допущена ошибка. \\

Мы попробовали изменить поле $e_7$ следующим образом: перенести $m$ из знаменателя в числитель и сделать его равным нулю. \\

Тогда поле станет выглядеть так: $e_7 = (z_1, 2z_2, 3z_3, 3z_4).$ \\

Тогда $[e_6, e_7] = me_5 + e_6 = e_6 = \left(i \varepsilon, i \varepsilon z_1, i \varepsilon z_1^2, z_2 + (1/2)z_1^2 \right).$ \\

Остальные коммутаторы, в которых присутствует поле $e_7$ останутся правильными. \\

Для проверки коммутаторов и тождеств Якоби по полям были написаны программы на Maple: \\

- V1_DEFAULT_COMMUTATORS.mw — с исходными полями; \\

- V2_COMMUTATORS_M_IN_NUMERATOR.mw — в поле $e_7$ перенесли $m$ из знаменателя в числитель; \\

- V3_COMMUTATORS_M_IN_NUMERATOR_AND_ZERO.mw — в поле $e_7$ перенесли $m$ из знаменателя в числитель и приравняли к 0;

\pagebreak
\section{Список использованной литературы}

1. Крутских В. В., Лобода А. В. Компьютерные алгоритмы реализации 7-мерных алгебр Ли // Материалы Международной научно-технической конференции «Актуальные проблемы прикладной математики, информатики и механики». Воронеж: ПМИМ ВГУ, 2022.

2. Лобода А. В., Атанов А. В., Албуткина П. Е. Об алгоритмах описания однородных подмногообразий многомерных пространств // Материалы XXVI международной научно-практической конференции ИПМТ. Воронеж, 2024. С. 380–391.

3. Атанов А. В. Орбиты разложимых 7-мерных алгебр Ли с sl(2)-подалгеброй // Уфимский математический журнал. 2022. Т. 14, № 1.

4. Акопян Р. С., Лобода А. В. О голоморфных реализациях пятимерных алгебр Ли // Алгебра и анализ. 2019. Т. 31, № 6. С. 1–37. 

5. Vu A. L., Nguyen T. A., Nguyen T. T. C., Nguyen T. T. M., Vo T. N. Classification of 7-dimensional solvable Lie algebras having 5-dimensional nilradicals // Communications in Algebra. 2023. Т. 51, № 5. С. 1866–1885.

6. Parry A. R. A Classification of Real Indecomposable Solvable Lie Algebras of Small Dimension with Codimension One Nilradicals // Магистерская диссертация. Логан, Юта, 2007.
\end{document}